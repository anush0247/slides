\documentclass[xcolor=dvipsnames]{beamer}

\usetheme{Frankfurt}
\usepackage{listings}
\usepackage{paralist}
\usecolortheme[RGB={0,104,139}]{structure}%deepskyblue
%\usecolortheme[named=Maroon]{structure}

\title[Cloud based IT Infra with Central Identity]{Cloud based IT Infra with Central Identity}
\subtitle{\{Project reboot\} - Phase I  }

\author{ \underline{Project Guide} \\ \hspace{2mm} \\ \small{ T. Chandra Shekar }  }
\institute{ \underline{Presenting by} \\ \hspace{2mm} \\ \textit {Team r3b00+ }  \\ \hspace{4mm} \\ \textit{Dept. of CSE, RGUKT - Nuzvid}}
\usepackage{color}
 
\definecolor{codegreen}{rgb}{0,0.6,0}
\definecolor{codegray}{rgb}{0.5,0.5,0.5}
\definecolor{codepurple}{rgb}{0.58,0,0.82}
\definecolor{backcolour}{rgb}{0.95,0.95,0.92}
\lstdefinestyle{mystyle}{
    backgroundcolor=\color{backcolour},   
    commentstyle=\color{codegreen},
    keywordstyle=\color{magenta},
    numberstyle=\tiny\color{codegray},
    stringstyle=\color{codepurple},
    basicstyle=\footnotesize,
    breakatwhitespace=false,         
    breaklines=true,                 
    captionpos=b,                    
    keepspaces=true,                 
    numbers=left,                    
    numbersep=5pt,                  
    showspaces=false,                
    showstringspaces=false,
    showtabs=false,                  
    tabsize=2
}
 
\lstset{style=mystyle}
 
\begin{document}


\begin{frame}
\titlepage
\end{frame}

\section{About us}
\begin{frame}{About us}

\small
\begin{center}
We are from team \textit{r3b00+}  \{reboot\} \\ \hspace{4cm} \\
\begin{tabular}{l  l }
T. Aneesh Kumar & N090247   \\
P. Nageswarao  & N091030  \\
P. Anesh  & N090977 \\
P. Jyothi Ram & N090990 \\
K. Naresh Chowdary  & N090331 \\
N. Venkata Sateesh  & N090935 \\
M. Sanyasi Rao & N090891  
\end{tabular}


\end{center}


\end{frame}
 
\setcounter{page}{1}
\section{Objective}
\begin{frame}{Objective}

The main objective of ``Cloud based IT Infra with Central Identity'' is to utilize exisiting hardware, turn them into private clouds and access all of its services using 
Central Identity, which can be available to third party developers as API with dynamic role management and service endpoints. \newline

New private cloud based IT Infra is aimed to develop using some opensource tools like OpenStack, NFS, LDAP, Ubuntu and etc \newline

Expecting to serve with high computational virtual machines to the research, academic, learning purpose, virtal labs rather than dedicated lab hardware.
\end{frame}

\section{Motivation}
\begin{frame}{Motivation}

\begin{itemize}
	\item No Central Identity, Central Storage \& High capacity hardware resource pool.
	\item Failed to maintain large user load web services like ONB, Exam servers, etc.
	\item Dedicated computer course labs like Matlab, VLSI, etc.
	\item No proper Web Application Security \& Standards.
	\item Inadequate resource requirements for Research.
\end{itemize}

\end{frame}



\section{Proposed System}
\subsection{Users \& IT Services}
\begin{frame}{Users \& IT Services}
We are grouping all IT Services that are required for University into one and identifing the user who will going to use them. All Users are catagorized into 4 groups $ ^{[1]}$
	
	\begin{itemize}
	\item Studens, Developers, Staff, faculty \& Researches
	\end{itemize}
\begin{figure}[H]
\begin{center}
\includegraphics[width=5cm]{./it.png}
\caption{ Simplified structure of the main users of IT services. \label{fig:Simplified structure of the main users of IT services. }}
\end{center}
\end{figure}
	
\end{frame}
\subsection{Cloud Infrastructures}
\begin{frame}{Cloud Infrastructures}
All University IT Services are deployed in a private cloud, constructed over exsiting infrastructure, that can be browdly viewed as 
	
\begin{figure}[H]
\begin{center}
\includegraphics[width=7cm]{./it2.png}
\caption{ IT Services and Users in Cloud Computing\label{fig:IT Services and Users in Cloud Computing }}
\end{center}
\end{figure}	
	
	
\end{frame}

\begin{frame}{Proposed System -  Main Components}
\begin{itemize}
	
	\item Network Components 
	\begin{itemize}
		\item AAA, LDAP, NFS
	\end{itemize}
	\item Central Identiy 
	\begin{itemize}
		\item Single Sign on
		\item Fedarated Identity
		\item Dynamic Role Based Access Control
		\item REST API to third party
	\end{itemize}
	\item Cloud Infrastructure
	\begin{itemize}
		\item Cloud Computing, Private Cloud, Open source tools
	\end{itemize}
\end{itemize}
\end{frame}

\section{Single Sign-On}
\subsection{What is SSO}
\begin{frame}{What is Single Sign-On?}
 
\begin{itemize}
	\item Single Sign-On (SSO) is an authentication process that allows a user to access multiple applications with one set of login credentials. 
	\item One login. All of RGUKT.
\end{itemize}

\begin{figure}
\includegraphics[width=8cm,height=3.5cm]{SSO}
\caption{Google SSO and Application \label{fig:Google SSO and Application}}
\end{figure}

\end{frame}
\subsection{Why SSO}
\begin{frame}{Why Single Sign-On?}

\begin{itemize}
	\item Signing Up everytime is troublesome
	\item I am a nerd, I can't remember all the passwords across multiple Apps.
	\item Is it possible to just sign-on once to perform all the actions?
	\item Is basic Authorization on many sites is secure? Any alternative?
	\item Yes!! Single sign-on can be used to answer all these Questions.
\end{itemize}

\end{frame}

\subsection{Advantages}
\begin{frame}{Advantages}
\begin{itemize}
	\item Ease burden on developers
	\item Improved user experience, no password lists to carry. Thus, improving productivity.
	\item Ease of Access through a single Central Database.
	\item Transfer of Sensitive Data across network is minimized.
	\item Enables users to login quickly and securely to all their applications.
	\item Auditing \& Statistical history reviewing simplified.
\end{itemize}
\end{frame}

\subsection{How well we will implement?}
\begin{frame}{How well we will implement?}
\begin{itemize}
\item We want to develope well structured and documented REST API
\item Designing neat and user friendly interface with Semantic UI
\item Technologies to be Used
\begin{figure}
\includegraphics[width=9.5cm,height=1.5cm]{technologies}
\end{figure}
\item Standards to be followed
\begin{figure}
\includegraphics[width=7cm,height=2cm]{standards}
\end{figure}
\end{itemize}
\end{frame}

\subsection{What is JSON?}
\begin{frame}{What is JSON?}

\textbf{JSON}\\

\begin{itemize}
\item JSON stands for \textbf{J}ava\textbf{S}cript \textbf{O}bject \textbf{N}otation
\item An open standard format that uses plain text to transmit data.
\item Used primarily to transmit data between a server and web application, as an alternative to XML.

\end{itemize}
\textbf{Example}
\lstinputlisting[language=Java,caption=JSON Example]{abc.json}
\end{frame}

\subsection{What is REST?}
\begin{frame}{What is REST?}

\textbf{REST API}

\begin{itemize}
\item REST stand for \textbf{RE}presentational \textbf{S}tate \textbf{T}ransfer
\item A Collection of simple URIs, and HTTP calls to those URIs and some JSON resources
\item Basic CRUD Operations
\end{itemize}

\begin{figure}
\includegraphics[width=7cm,height=2cm]{CRUD}
\caption{CRUD Operations \label{fig:CRUD Operations}}
\end{figure}

\end{frame}


\subsection{What is REST? (Contd...)}
\begin{frame}{REST API Example}
\textbf{Syntax}
\begin{itemize}
	\item http://it-ebooks-api.info/v1/book/:id/:author/
\end{itemize}
\textbf{Example}
\begin{itemize}
	\item Request URI -- http://it-ebooks-api.info/v1/book/1234/
	\item Response 
\end{itemize}
\lstinputlisting[language=Java,caption=REST Example]{abc2.json}

\end{frame}

\section{RBAC}
\subsection{Introduction}
\begin{frame}{Introduction to RBAC}
\begin{itemize}
 \item Role Based Access Control(RBAC) assigns users to roles and then roles to permissions, It solves problems of least privilege, separation of duty and other security issues
 \item In RBAC model, these rights are defined based on the role that individuals are assigned to in an organization
 \item It overcomes the problems in DAC which is flexible but not secure and MAC which is Secure but not flexible
\end{itemize}
\begin{figure}[H]
\includegraphics[width=9cm,height=3cm]{sunny1}
\caption{Scenario of RBAC\label{fig:Scenario of RBAC}}
\end{figure}
\end{frame}

\subsection{Idea of RBAC}
\begin{frame}{Basic Idea of RBAC}
\begin{itemize}
	\item  Access Control policy is embodied in various components of RBAC such as,
	\begin{itemize}
	\item Role-Permission relationships
	\item User-Role relationships
	\item Role-Role relationships
	\end{itemize}
	\item Users get roles corresponding permissions by getting roles to operate on the objects
	\item RBAC model is defined in terms of three model components - Core RBAC, Hierarchical RBAC and Constraint RBAC
\end{itemize}
\begin{figure}[H]
\includegraphics[width=10cm,height=3cm]{sunny2_r}
\end{figure}
\end{frame}
\subsection{Structure of RBAC}
\begin{frame}{Structure Diagram of RBAC Model}
\begin{columns}
 
\column{0.6\textwidth}
\begin{itemize}
\item The Structure diagram of role based access control model consists of role hierarchies and constraints
\item Role hierarchical relationship expresses the inheritance in roles permissions
\begin{itemize}
\item User inheritance
\item Permission inheritance
\item Activation inheritance
\end{itemize}
\item Constraints in RBAC adds separation of duty relations
\begin{itemize}
\item Mutual exclusion
\item Pre-condition
\item Cardinality
\end{itemize}
\end{itemize}




 
\column{0.4\textwidth}
\begin{figure}[H]
\includegraphics[width=4cm,height=3cm]{sun3}
\caption{RBAC3 Mode}
\end{figure}
\end{columns}
\end{frame}

\subsection{Dynamic RBAC}
\begin{frame}{Dynamic RBAC}


Dynamic RBAC overcomes the shortages of the traditional RBAC by adding with dynamic constraints and  permissions
\begin{itemize}
 \item It retains original static constraints of traditional RBAC
 \item The App creator no need to go for administration, himself he can add or create a role for users
 \item It supports each user has different levels of permission at different time
\end{itemize}
\begin{figure}[H]
\includegraphics[width=10cm,height=3cm]{ass}
\caption{Different Types of Association}
\end{figure}
\end{frame}
\section{Cloud Computing}
\begin{frame}{Cloud Computing - Definition }

What is Cloud Computing ...? \\
\hspace{4cm} \\
\textit{``Cloud computing is a model for enabling convenient, on-
demand network access to a shared pool of configurable
computing resources (e.g., networks, servers, storage,
applications, and services) that can be rapidly provisioned
and released with minimal management effort or service
provider interaction''} $ ^{[1]} $

\end{frame}

\begin{frame} {Cloud Computing - Characteristics }
One can define Cloud Computing with essential characteristics like

\begin{itemize}
\item On-demand self-service
\item Broad network access
\item Resource pooling
\item Rapid elasticity
\item Measured Service
\end{itemize}
\end{frame}

\begin{frame}{Cloud Computing - Servcice Models }

If we providing any thing as a service comes, that will comes into Cloud Computing. Various Service Delivery Models listed bellow.

\begin{itemize}
\item Software as a Service (SaaS) 
%\begin{itemize}
%	\item Google Docs, Photo editors, Calculators, etc
%\end{itemize}
\item Platform as a Service (PaaS)
%\begin{itemize}
%	\item Harkoo, Aneka, Google App engine, etc
%\end{itemize}
\item Infrastructure as a Service (IaaS)
%\begin{itemize}
%	\item AWS, Openstack, Cloudstack, Opennebula, Eucalyptus etc
%\end{itemize}
%\item Data storage as a Service (DaaS)
%\begin{itemize}
%	\item Amazon S3, DropBox, SkyDrive, etc
%\end{itemize}
\item Anything as a Service (Xaas)
\end{itemize}
\begin{figure}[H]
 \centering
 \includegraphics[width=8cm]{./service.png}
 \caption{Cloud Computing - Servcice Models \label{fig:Cloud Computing - Servcice Models} }
\end{figure}

\end{frame}

\begin{frame}{Cloud Computing - Deployment Model }

We can deploy the cloud in various ways.

\begin{itemize}
\item Public Cloud
\item Private Cloud
\item Hybrid cloud
\end{itemize}

\begin{figure}[H]
 \centering
 \includegraphics[width=5cm]{./model.png}
 \caption{Cloud Computing - Deployment Models \label{fig:model} }
\end{figure}
\end{frame}

\section{Private Clouds}
\begin{frame}{Private Clouds -- Introduction}

As per our concern we mainly focused about private clouds inorder to ensure Organizational data security \& High resource utilization

\hspace{4cm}\\

\textbf{``Private Cloud''} \\ 
\hspace{16mm} -- \textit{It is one of the cloud deployment model where the resources of small or medium organization are united and cattered to users of the that organization or outsourced through internet.} 

\end{frame}

\begin{frame}{Private Clouds -- Open Soruce Tools}

We can construct private cloud using some open source tools like Openstack, Cloudstack, OpenNebula. \\

\hspace{4cm} 

We can use this private cloud to deploy various services like Departmental Websites, Notice Boards, Events portal, High Computational Virtual Machines for Virtual Labs, High Performance Computing, Big data analytics.
\begin{figure}[H]
 \centering
 \includegraphics[width=5cm]{./cloud.jpg}
 \caption{Private Cloud - Open source tools \label{fig:cloud} }
\end{figure}

\end{frame}

\section{Questions?}
\begin{frame}{Questions?}
\centering
Questions ?
\end{frame}


\begin{frame}{One word to say ..}
Thank you
\end{frame}

\end{document}