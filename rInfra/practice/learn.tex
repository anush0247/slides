\documentclass[xcolor=dvipsnames]{beamer}
%\usetheme{rose}
\usetheme{Frankfurt}
%\usetheme{Berlin}
%\usetheme{rose}
\usecolortheme[RGB={0,104,139}]{structure}
%\usecolortheme{seahorse}
%\usecolortheme{rose}
\title{My PPT Title}
\subtitle{my sub title}

\author[Aneesh]{its me}
\institute[Universities of Rijeka and Berlin]{
\inst{1}Department of Informatics\\
University of Rijeka
\and
\inst{2}Fakult\"at f\"ur Elektrotechnik und Informatik\\
Technical University of Berlin}


\begin{document}
\begin{frame}
\titlepage
\end{frame}
\begin{frame}
\begin{itemize}
\item<1-> Eggs
\item<2-> Plants
\note[item]<2>{Tell joke about plants.}
\note[item]<2>{Make it short.}
\item<3-> Animals
\end{itemize}
\end{frame}


\begin{frame}
\begin{itemize}[<+-| alert@+>]
\item Apple
\item Peach
\item Plum
\item Orange
\end{itemize}
\end{frame}
\begin{frame}[<+->]
\begin{theorem}
$A = B$.
\end{theorem}
\begin{proof}
\begin{itemize}
\item Clearly, $A = C$.
\item As shown earlier,
\item<3-> Thus $A = B$.
\end{itemize}
\end{proof}
\end{frame}
\section{abc}
\begin{frame}
\frametitle{What Are Prime Numbers?}
\begin{definition}
A \alert{prime number} is a number that has exactly two divisors.
\end{definition}
\pause
\begin{example}
\begin{itemize}
\item 2 is prime (two divisors: 1 and 2).
\pause
\item 3 is prime (two divisors: 1 and 3).
\pause
\item 4 is not prime (\alert{three} divisors: 1, 2, and 4).
\end{itemize}
\end{example}
\end{frame}
\section{abc2}
\begin{frame}
\frametitle{There Is No Largest Prime Number}
\framesubtitle{The proof uses \textit{reductio ad absurdum}.}
\begin{theorem}
There is no largest prime number.
\end{theorem}
\begin{proof}
\begin{enumerate}
\item<1-> Suppose $p$ were the largest prime number.
\item<2-> Let $q$ be the product of the first $p$ numbers.
\item<3-> Then $q + 1$ is not divisible by any of them.
\item<1-> But $q + 1$ is greater than $1$, thus divisible by some prime
number not in the first $p$ numbers.\qedhere
\end{enumerate}
\end{proof}
\uncover<4->{The proof used \textit{reductio ad absurdum}.}
\end{frame}

\begin{frame}[allowframebreaks,allowdisplaybreaks]
\frametitle{What’s Still To Do?}
\begin{block}{Answered Questions}
How many primes are there?
\end{block}
\begin{block}{Open Questions}
Is every even number the sum of two primes?
\end{block}
\begin{block}{Answered Questions}
How many primes are there?
\end{block}
\begin{block}{Open Questions}
Is every even number the sum of two primes?
\end{block}
\begin{block}{Answered Questions}
How many primes are there?
\end{block}
\begin{block}{Open Questions}
Is every even number the sum of two primes?
\end{block}
\end{frame}

\begin{frame}
\frametitle{What’s Still To Do?}
\begin{itemize}
\item Answered Questions
\item How many primes are there?
\item Open Questions
\begin{itemize}
\item Is every even number the sum of two primes?
\end{itemize}
\end{itemize}
\begin{description}
\item[Lion] King of the savanna.
\item[Tiger] King of the jungle.
\end{description}
\structure{Paragraph Heading.}
\end{frame}

\begin{frame}
\frametitle{What’s Still To Do?}
\begin{columns}
\column{.5\textwidth}
\begin{block}{Answered Questions}
How many primes are there?
\end{block}
\column{.5\textwidth}
\begin{block}{Open Questions}
Is every even number the sum of two primes?
\end{block}
\end{columns}
\begin{block}{Open Questions}
Is every even number the sum of two primes?
\cite{Goldbach1742}
\end{block}

\begin{beamercolorbox}[ht=2.5ex,dp=1ex,center]{title in head/foot}
\usebeamerfont{title in head/foot}
\insertshorttitle
\end{beamercolorbox}%
\begin{beamercolorbox}[ht=2.5ex,dp=1ex,center]{author in head/foot}
\usebeamerfont{author in head/foot}
\insertshortauthor
\end{beamercolorbox}
\end{frame}


\begin{frame}[fragile]
\frametitle{An Algorithm For Finding Primes Numbers.}
\begin{verbatim}
int main (void)
{
std::vector<bool> is_prime (100, true);
for (int i = 2; i < 100; i++)
if (is_prime[i])
{
std::cout << i << " ";
for (int j = i; j < 100; is_prime [j] = false, j+=i);
}
return 0;
}
\end{verbatim}
\begin{uncoverenv}<2>
Note the use of \verb|std::|.
\end{uncoverenv}
\end{frame}



\begin{frame}{References}
\begin{thebibliography}{10}
\bibitem{Goldbach1742}[Goldbach, 1742]
Christian Goldbach.
\newblock A problem we should try to solve before the ISPN ’43 deadline,
\newblock \emph{Letter to Leonhard Euler}, 1742.
\bibitem{Goldbach1743}[Goldbach, 1742]
Christian Goldbach.
\newblock A problem we should try to solve before the ISPN ’43 deadline,
\newblock \emph{Letter to Leonhard Euler}, 1742.
\end{thebibliography}
\end{frame}



\end{document}
